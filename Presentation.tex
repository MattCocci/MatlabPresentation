%%%%%%%%%%%%%%%%%%%%%%%%%%%%%%%%%%%%%%%%%%%%%%%%%%%%%%%%%%%%%%%%%%%%%
%% Configure File %%%%%%%%%%%%%%%%%%%%%%%%%%%%%%%%%%%%%%%%%%%%%%%%%%%
%%%%%%%%%%%%%%%%%%%%%%%%%%%%%%%%%%%%%%%%%%%%%%%%%%%%%%%%%%%%%%%%%%%%%

\documentclass{beamer}
%\documentclass[serif]{beamer} % For serif latex font
\usepackage{lmodern}
\usetheme{Boadilla}
\usepackage{upquote}
%\usetheme{Frankfurt}

% Set up the colors to use with "alerted" text
% --> See the slide "Highlighting Certain Points?"
\setbeamercolor{dull text}{fg=gray, bg=}
\setbeamercolor{alerted text}{fg=black, bg=}

\usepackage{courier}
\usepackage{color}
\definecolor{codegreen}{RGB}{28,172,0}
\definecolor{codelilas}{RGB}{170,55,241}

\usepackage{listings}
    %   For including code snippets written directly in this doc
\lstdefinestyle{bash}{%
  language=bash,%
  basicstyle=\footnotesize\ttfamily,%
  showstringspaces=false,%
  commentstyle=\color{gray},%
  keywordstyle=\color{blue},%
  xleftmargin=0.0in,%
  xrightmargin=0.0in
}

\lstdefinestyle{matlab}{%
  language=Matlab,%
  basicstyle=\scriptsize\ttfamily,%
  breaklines=true,%
  morekeywords={matlab2tikz},%
  keywordstyle=\color{blue},%
  morekeywords=[2]{1}, keywordstyle=[2]{\color{black}},%
  identifierstyle=\color{black},%
  stringstyle=\color{codelilas},%
  commentstyle=\color{codegreen},%
  showstringspaces=false,%
    %   Without this there will be a symbol in
    %   the places where there is a space
  emph=[1]{for,end,break,switch,case},emphstyle=[1]\color{blue},%
    %   Some words to emphasise
}

\lstdefinestyle{matlabNoBlue}{%
  language=Matlab,%
  basicstyle=\scriptsize\ttfamily,%
  breaklines=true,%
  morekeywords={matlab2tikz},%
  keywordstyle=\color{black},%
  morekeywords=[2]{1}, keywordstyle=[2]{\color{black}},%
  identifierstyle=\color{black},%
  stringstyle=\color{codelilas},%
  commentstyle=\color{codegreen},%
  showstringspaces=false,%
    %   Without this there will be a symbol in
    %   the places where there is a space
  emph=[1]{for,end,break,switch,case},emphstyle=[1]\color{blue},%
    %   Some words to emphasise
}


%%%%%%%%%%%%%%%%%%%%%%%%%%%%%%%%%%%%%%%%%%%%%%%%%%%%%%%%%%%%%%%%%%%%%
%% Fixup Code %%%%%%%%%%%%%%%%%%%%%%%%%%%%%%%%%%%%%%%%%%%%%%%%%%%%%%%
%%%%%%%%%%%%%%%%%%%%%%%%%%%%%%%%%%%%%%%%%%%%%%%%%%%%%%%%%%%%%%%%%%%%%

% This next line removes the institute parens in the
    % footer
\defbeamertemplate*{footline}{my infolines theme}
{%
  \leavevmode%
  \hbox{%
  \begin{beamercolorbox}[wd=.333333\paperwidth,ht=2.25ex,dp=1ex,center]{author in head/foot}%
    \usebeamerfont{author in head/foot}\insertshortauthor~\insertshortinstitute%
  \end{beamercolorbox}%
  \begin{beamercolorbox}[wd=.333333\paperwidth,ht=2.25ex,dp=1ex,center]{title in head/foot}%
    \usebeamerfont{title in head/foot}\insertshorttitle
  \end{beamercolorbox}%
  \begin{beamercolorbox}[wd=.333333\paperwidth,ht=2.25ex,dp=1ex,right]{date in head/foot}%
    \usebeamerfont{date in head/foot}\insertshortdate{}\hspace*{2em}
    \insertframenumber{} / \inserttotalframenumber\hspace*{2ex}
  \end{beamercolorbox}}%
}



%%%%%%%%%%%%%%%%%%%%%%%%%%%%%%%%%%%%%%%%%%%%%%%%%%%%%%%%%%%%%%%%%%%%%
%% Title Slide %%%%%%%%%%%%%%%%%%%%%%%%%%%%%%%%%%%%%%%%%%%%%%%%%%%%%%
%%%%%%%%%%%%%%%%%%%%%%%%%%%%%%%%%%%%%%%%%%%%%%%%%%%%%%%%%%%%%%%%%%%%%

\begin{document}
\title{\textsc{matlab }\\ Practices with Examples}
\author{Matt Cocci \& Sara Shahanaghi}
\date{}

\frame{\titlepage}


%%%%%%%%%%%%%%%%%%%%%%%%%%%%%%%%%%%%%%%%%%%%%%%%%%%%%%%%%%%%%%%%%%%%%
%% Table of Contents %%%%%%%%%%%%%%%%%%%%%%%%%%%%%%%%%%%%%%%%%%%%%%%%
%%%%%%%%%%%%%%%%%%%%%%%%%%%%%%%%%%%%%%%%%%%%%%%%%%%%%%%%%%%%%%%%%%%%%

% Make table of contents slide
    % Can use sections and subsections before new
    % frames/slides which will get put in Table of Contents
\frame{%
    \frametitle{Topics and Overview}
    \tableofcontents
}


%%%%%%%%%%%%%%%%%%%%%%%%%%%%%%%%%%%%%%%%%%%%%%%%%%%%%%%%%%%%%%%%%%%%%
%% Presentation %%%%%%%%%%%%%%%%%%%%%%%%%%%%%%%%%%%%%%%%%%%%%%%%%%%%%
%%%%%%%%%%%%%%%%%%%%%%%%%%%%%%%%%%%%%%%%%%%%%%%%%%%%%%%%%%%%%%%%%%%%%

\lstset{style=matlab}
\section{Data Structures}

  \begin{frame}[fragile]
    \frametitle{Data Structures}

    Main types
    \begin{enumerate}
      \item Numeric Arrays
        \begin{lstlisting}
          1:10 [1 2 3 4] ones(4,5) nan(100,2,5)
        \end{lstlisting}
      \item Cell Arrays: Heterogeneous(-normative) information
        \begin{lstlisting}
  {'jerry', 'elaine', 'kramer', 'george'}
  {1, 'does', @(x) ~x, {'simply', 'walk', 'into Mordor'}}
        \end{lstlisting}
      \item Structs and Struct Arrays: \emph{Named} heterogeneous
        information
        \begin{lstlisting}
          data.series = {'GDPGrowth', 'PCEGrowth'};
          data.frqcy  = {'q', 'm'};
          data.values = randn(T,2);
        \end{lstlisting}
    \end{enumerate}

	\end{frame}

\subsection{Numeric Arrays}
  \begin{frame}[fragile]
    \frametitle{Numeric Arrays: Liberal Use of Pre-Allocation and \text{NaN}s}

    Always initialize numeric arrays before loops, often to
    \texttt{NaN}:
    \begin{lstlisting}
      toPlot = nan(T,Nseries);
      for t = 1:T
        toPlot(t) = rand(Nseries,1);
      end
    \end{lstlisting}
    Why?
    \begin{enumerate}
      \item Speeds up the code
      \item Forces you to \emph{name} sizes and dimensions
      \item Plotting \texttt{NaN}'s is convenient
    \end{enumerate}

	\end{frame}

  \begin{frame}[fragile]
    \frametitle{Numeric Arrays: Liberal Use of Pre-Allocation and \text{NaN}s}

    \begin{lstlisting}
      1947   1   1       NaN      NaN
      1947   2   1    8.1078      NaN
      ...               ...      ...
      2015   4   1    1.2388  0.47595
      2015   5   1    5.4667   3.8075
    \end{lstlisting}
    \pause
    %\begin{figure}
      %\includegraphics[trim={2cm, 5cm, 2cm, 7cm}, clip, scale=0.35]{Examples/Ex1.pdf}
    %\end{figure}
	\end{frame}

  \begin{frame}[fragile]
    \frametitle{Numeric Arrays: Liberal Use of Pre-Allocation and \text{NaN}s}

    Using \texttt{NaN}'s:
    \begin{enumerate}
      \item Sizes are standard, transparent, unchanging.
      \item Placement in plots is more precise.
    \end{enumerate}
    %\begin{figure}
      %\includegraphics[trim={2cm, 5cm, 2cm, 7cm}, clip, scale=0.35]{Examples/Ex1.pdf}
    %\end{figure}
	\end{frame}



  \begin{frame}[fragile]
    \frametitle{Numeric Arrays: Multi-Dimensional Panels}

    Examples of multi-demensional arrays:
    \begin{enumerate}
      \item Panel data
        \begin{lstlisting}
        nan(Nindividuals,Ncovariates,T)
        \end{lstlisting}
      \item Vintage data
        \begin{lstlisting}
        nan(Nobsdates,Nseries,Nvintages)
        \end{lstlisting}
      \item SEP Data
        \begin{lstlisting}
        nan(Npart,Nhorizons,Nvintages,Nvar)
        \end{lstlisting}
    \end{enumerate}
	\end{frame}


  \begin{frame}[fragile]
    \frametitle{Numeric Arrays: Multi-Dimensional Panels}

    Reshaping
    \begin{enumerate}
      \item \texttt{reshape(X,[M N K])}
      \item \texttt{reshape(X,M,N,K)}
      \item \texttt{reshape(X,M,[],K)}
    \end{enumerate}
	\end{frame}


  \begin{frame}[fragile]
    \frametitle{Numeric Arrays: Multi-Dim. Arrays, the Annoying Part}

    Suppose
    \begin{lstlisting}
      X = nan(3,3,3);
      X(:,:,1) = 1;
      X(:,:,2) = 2;
      X(:,:,3) = 3;
    \end{lstlisting}\pause
    \begin{lstlisting}
      >> X
      X(:,:,1) =
          1     1     1
          1     1     1
          1     1     1
      X(:,:,2) =
          2     2     2
          2     2     2
          2     2     2
      X(:,:,3) =
          3     3     3
          3     3     3
          3     3     3
    \end{lstlisting}\pause
    Will attempt \texttt{X(1,:,:)}
	\end{frame}


  \begin{frame}[fragile]
    \frametitle{Numeric Arrays: Multi-Dim. Arrays, the Annoying Part}

    \begin{enumerate}
      \item Without \texttt{squeeze}
        \begin{lstlisting}
        >> X(1,:,:)
        ans(:,:,1) =
            1     1     1
        ans(:,:,2) =
            2     2     2
        ans(:,:,3) =
            3     3     3
        \end{lstlisting}

      \item Check the size
        \begin{lstlisting}
        >> size(X(1,:,:))
        ans =
            1     3     3
        \end{lstlisting}\pause

      \item \texttt{squeeze(X(1,:,:))}\pause
        \begin{lstlisting}
          >> squeeze(X(1,:,:))
          ans =
              1     2     3
              1     2     3
              1     2     3
      \end{lstlisting}
    \end{enumerate}
	\end{frame}


  \subsection{Cell Arrays}

  \begin{frame}[fragile,shrink=10]
    \frametitle{Cell Arrays: Easier String Handling}

    Working with strings
    \begin{enumerate}
      \item A string is a type \texttt{char} array; each letter is an element
      \begin{lstlisting}
        >> class('someString')
        ans =
          char
      \end{lstlisting}\pause

      \item When you want to store an \emph{array} of strings (like a
      date array), you could stack them:
      \begin{lstlisting}
        >> ['2015Q1'; '2015Q2']
          ans =
          2015Q1
          2015Q2
      \end{lstlisting}\pause
      Dumb in practice
      \begin{lstlisting}
        >> ['abc'; 'defg']
        Error using vertcat
        Dimensions of matrices being concatenated are not consistent.
      \end{lstlisting}\pause

      \item Use cell arrays instead:
      \begin{lstlisting}
        >> {'abc'; 'defg'}
          ans =
              'abc'
              'defg'
      \end{lstlisting}
    \end{enumerate}
    Store series names, data headers, date strings, and arrays of
    strings in cells.

	\end{frame}



  \begin{frame}[fragile]
    \frametitle{Cell Arrays: Holding and Unpacking Function Arguments}

    Standard way to pass arguments to a function:
    \begin{lstlisting}
      plot(x, y, 'Color', 'blue', 'LineWidth', 2, 'Marker', 'x')
    \end{lstlisting}\pause
    Another way
    \begin{lstlisting}
      args = {'Color', 'blue', 'LineWidth', 2, 'Marker', 'x'};
      plot(x, y, args{:})
    \end{lstlisting}
    or to only use a subset
    \begin{lstlisting}
      plot(x, y, args{1:2})
    \end{lstlisting}
	\end{frame}

  \begin{frame}[fragile]
    \frametitle{Cell Arrays: Holding and Unpacking Function Arguments}

    Why in the world would you ever want to do this?
    \begin{enumerate}
      \item Could group sets of options intelligently
\begin{lstlisting}
markerOptions = {'Marker', 'x', 'MarkerSize', 5};
lineOptions = {'LineWidth', 2, 'Color', 'blue'};

plot(x1,y1,markerOptions{:})
plot(x2,y2,lineOptions{:})
plot(x3,y3,lineOptions{:},markerOptions{:})
\end{lstlisting}\pause
      \item You can define options at the top of a program, rather than
        burying them in the body of a for loop
\begin{lstlisting}
  lgdTitles  = {'Donald', 'Trump'};
  lgdOptions = {'FontSize', 16, 'Location', 'SouthEast'};
  for n = 1:ever
    % Tale of Two Cities, reproduced in comments
    % ...
    % 10000000 lines later, stuff gets plotted in gold

    lgd = legend(lgdTitles{:});
    set(lgd, lgdOptions{:});
  end
\end{lstlisting}
    \end{enumerate}

	\end{frame}


  \begin{frame}[fragile]
    \frametitle{Aside: Varargin}

    You can define a function to take variable arguments
    \begin{lstlisting}
      function [fig] = myplot(x,y,varargin)

        plot(x,y,varargin{:})

      end

      myplot(x,y)
      myplot(x,y,'LineWidth',5)
    \end{lstlisting}
    Variable arguments passed to \texttt{myplot} will be stored in a
    cell named \texttt{varargin}.

    \vspace{10pt}
    More on varargin and functional programming below.
	\end{frame}

\subsection{Structs and Struct Arrays}

  \begin{frame}[fragile]
    \frametitle{Structs: Named Heterogeneous Information}

    Great for creating datasets and grouping metadata with numerical data
    \begin{lstlisting}
      d = csvread('data.txt');
      mydata.values = d(:,2:end);
      mydata.dates  = datenum(d(:,1));
      mydata.colums = {'GDP', 'PCE', 'CPI'};
    \end{lstlisting}\pause
    \begin{lstlisting}
      plot(mydata.dates, mydata.values)
      legend(mydata.columns{:})
    \end{lstlisting}
    Can also pass information as a \emph{bundle} to functions, rather
    than as an unpacked list of arguments
    \begin{lstlisting}
      plotData(mydata)

      % Rather than
      plotData(dates, values, titles, transformation, colors)
    \end{lstlisting}
	\end{frame}



  \begin{frame}[fragile]
    \frametitle{Structs: Struct Arrays by Example}

    What a typical struct looks like:
    \lstset{style=matlabNoBlue}
    \begin{lstlisting}
    >> fred.latest('GDPC1')
    ans =
            info: [1x1 struct]
          series: 'GDPC1'
       frequency: 'Q'
           units: 'lin'
          pseudo: NaN
        realtime: 736158
            date: [273x1 double]
           value: [273x1 double]
    \end{lstlisting}\pause
    \lstset{style=matlab}
    But this has a dimension---it's 1$\times$1
	\end{frame}


  \begin{frame}[fragile]
    \frametitle{Structs: Struct Arrays by Example}

      So you can stack them and get an an array of structs:
      \lstset{style=matlabNoBlue}
      \begin{lstlisting}
      >> fred.latest({'GDPC1', 'CPIAUCSL'},0)
      ans =
      2x1 struct array with fields:
          info
          series
          frequency
          units
          pseudo
          realtime
          date
          value
      \end{lstlisting}\pause
      \begin{lstlisting}
      >> ans(1)
      ans =
              info: [1x1 struct]
            series: 'GDPC1'
         frequency: 'Q'
             units: 'lin'
            pseudo: NaN
          realtime: 736158
              date: [273x1 double]
             value: [273x1 double]
      \end{lstlisting}
      \lstset{style=matlab}
	\end{frame}

  \begin{frame}[fragile]
    \frametitle{Structs: Struct Arrays by Example}

    Punchline
    \begin{enumerate}
      \item Structs are nice.
      \item Struct arrays are nicer.
      \item They let have an \emph{array} of bundled data and metadata,
        and you can create them as simply as
        \begin{lstlisting}
          structArray = [struct1; struct2];
        \end{lstlisting}
      \item You can loop over them
        \begin{lstlisting}
          for t = 1:T
            plot(   structArray(t).data  )
            legend( structArray(t).series)
            title(  structArray(t).title )
          end
        \end{lstlisting}
    \end{enumerate}
	\end{frame}

  \begin{frame}[fragile]
    \frametitle{Structs: Avoiding \texttt{eval}}

    Common Stata behavior when you want variables \texttt{x1},
    \texttt{x2}, \ldots
    \begin{lstlisting}
      local x`i' = 1
    \end{lstlisting}\pause
    Literal Matlab translation, that you will sometimes see
    \begin{lstlisting}
      eval(['x' num2str(i) '= 1])
    \end{lstlisting}\pause
    That will produce the string
    \begin{lstlisting}
      'x1 = 1'
    \end{lstlisting}
    and then evaluate it like you had typed it into the Matlab prompt.\pause

    \vspace{20pt}
    But \texttt{`eval'} is exactly why we can't have nice things.
	\end{frame}


  \begin{frame}[fragile]
    \frametitle{Structs: Evaling \texttt{avoid}}

    Better solution is to use structs
    \begin{lstlisting}
      for n = 1:N
        fldName = ['x' num2str(n)];
        mystruct.(fldName) = n;
      end
    \end{lstlisting}\pause
    Even better: Specify the names explicitly
    \begin{lstlisting}
      series = {'GDPC', 'PCE', 'CPI'}
      for n = 1:length(series)
        s = series{n};
        mystruct.(s) = transform(mystruct.(s));
      end
    \end{lstlisting}\pause
    Called ``dynamic referencing'' which is a pretentious name for
    ``looping over dictionary keys''
	\end{frame}


\section{Functional Programming}

\subsection{Functions vs. Scripts}
  \begin{frame}[fragile]
    \frametitle{Functional Programming}

    \begin{centering}
    \begin{center}
      \huge
      \textbf{Short functions not long scripts}
    \end{center}
    \end{centering}
	\end{frame}


  \begin{frame}[fragile]
    \frametitle{Functional Programming}

    Why short functions not long scripts?
    \begin{enumerate}
      \item Focus on inputs and outputs
      \item Easy to understand, maintain, and improve a part of the code
        without worrying about or screwing up the whole codebase\pause
      \item You parallelize functions, not scripts\pause
      \item Easy to error-proof functions\pause
      \item Easier to debug\pause
      \item Separate namespaces
    \end{enumerate}
	\end{frame}

\subsection{Namespaces}

  \begin{frame}[fragile]
    \frametitle{Functional Programming: Separate Namespaces}

    Define a function
    \begin{lstlisting}
      function [x] = myfcn(x)
        x = 2*x;
      end
    \end{lstlisting}\pause
    Variable definitions are local
    \begin{lstlisting}
      >> myfcn(5)
      ans =
          10
    \end{lstlisting}\pause
    \lstset{style=matlabNoBlue}
    \begin{lstlisting}
      >> x
      Undefined function or variable 'x'.
    \end{lstlisting}
    \lstset{style=matlab}
	\end{frame}


  \begin{frame}[fragile]
    \frametitle{Functional Programming: Separate Namespaces}

    Define a function
    \begin{lstlisting}
      function [x] = myfcn(x)
        x = 2*x;
      end
    \end{lstlisting}\pause
    There are no side effects
    \begin{lstlisting}
      >> x = 5;
      >> myfcn(x)
      ans =
          10
    \end{lstlisting}\pause
    \begin{lstlisting}
      >> x
      x =
          5
    \end{lstlisting}
	\end{frame}

  \begin{frame}[fragile]
    \frametitle{Functional Programming: Separate Namespaces}

    Separate namespaces limit side effects:
    \begin{lstlisting}
      function [returned] = myfcn(a,b,c,d)
        % Operations on a,b,c,d
      end
    \end{lstlisting}
    Features:
    \begin{enumerate}
      \item Every time you pass an argument to a function
      \begin{itemize}
        \item Matlab makes a copy of that input\pause
        \item Only the function sees that copy. The function acts on
          and modifies it.\pause
        \item When the function is done and returns something, it
          kills and sinks all copies of input arguments in the East
          River, never to be seen again, like a goddamn mob boss
      \end{itemize}
      \item Advantage: You can reuse variable names while ignoring and
        not impacting other parts of the code \emph{outside} the
        function.\pause
      \item Advantage: Narrow focus on inputs/outputs when coding
    \end{enumerate}
	\end{frame}


\subsection{Debugging}

  \begin{frame}[fragile]
    \frametitle{Functional Programming: Debugging}

    Placing breakpoints in your functions:
    \begin{lstlisting}
      function [returned] = myfcn(x)
        x = 2*x;
        keyboard
        returned = x;
      end
    \end{lstlisting}\pause
    Or, at the Matlab prompt:
    \begin{lstlisting}
      >> dbstop in myfcn at 2
      >> myfcn(2)
      2     x = 2*x;
      K>>
    \end{lstlisting}\pause
    Other useful commands
    \begin{enumerate}
      \item \texttt{dbstep}, \texttt{dbcont}\pause
      \item \texttt{dbup}, \texttt{dbdown}
    \end{enumerate}
	\end{frame}


  \begin{frame}[fragile]
    \frametitle{Functional Programming: \texttt{try}-\texttt{catch}}

    \texttt{try}-\texttt{catch} statements let you error-proof your
    code:
    \begin{lstlisting}
      try
        y = fcn(x);
      catch
        y = NaN;
      end
    \end{lstlisting}
	\end{frame}


\subsection{\texttt{feval}, \texttt{arrayfun}, \texttt{cellfun}}

  \begin{frame}[fragile]
    \frametitle{Functional Programming: Functions as Objects}

    Functions are (X-Men) first-class:
    \begin{enumerate}
      \item Functions take arguments\pause
      \item Functions can be \emph{passed as arguments}
    \end{enumerate}
    Example:
    \begin{lstlisting}
      >> x = 1:10;
      >> max(1:10)
      \end{lstlisting}\pause
      \vspace{-12pt}
      \begin{lstlisting}
      ans =
          10
      \end{lstlisting}\pause
      \vspace{-12pt}
      \begin{lstlisting}
      >> min(1:10)
      ans =
          1
    \end{lstlisting}
	\end{frame}


  \begin{frame}[fragile]
    \frametitle{Functional Programming: Functions as Objects}

    Another way
    \begin{lstlisting}
      >> x = 1:10;
      >> fcns = {'max', 'min'};
      >> feval(fcns{1}, x);
      \end{lstlisting}\pause
      \vspace{-12pt}
      \begin{lstlisting}
      ans =
          10
      \end{lstlisting}\pause
      \vspace{-12pt}
      \begin{lstlisting}
      >> feval(fcns{2}, x)
      ans =
          1
    \end{lstlisting}
	\end{frame}

  \begin{frame}[fragile]
    \frametitle{Functional Programming: Functions as Objects}

    With the \texttt{@} notation
    \begin{lstlisting}
      >> x = 1:10;
      >> fcns = {@max, @min};
      >> feval(fcns{1}, x);
      \end{lstlisting}
      \vspace{-12pt}
      \begin{lstlisting}
      ans =
          10
      \end{lstlisting}
      \vspace{-12pt}
      \begin{lstlisting}
      >> feval(fcns{2}, x)
      ans =
          1
    \end{lstlisting}
	\end{frame}


  \begin{frame}[fragile]
    \frametitle{Functional Programming: Functions as Objects}

    Why would you ever want to do this?\pause
    \vspace{10pt}

    Oftentimes, you want to loop over functions
    \begin{lstlisting}
      data = randn(1000,1); % <- Survey data
      fcns = {'mean', 'median', 'mode', 'min', 'max'};
    \end{lstlisting}\pause
    \begin{lstlisting}
      stats = struct();
      for n = 1:length(fcns)
        stats.(fcns{n}) = feval(fcns{n}, data);
      end
      \end{lstlisting}\pause
      \lstset{style=matlabNoBlue}
      \begin{lstlisting}
      stats =
          mean: -0.0326
        median: -0.0304
          mode: -3.2320
           min: -3.2320
           max: 3.5784
    \end{lstlisting}
    \lstset{style=matlab}
	\end{frame}

  \begin{frame}[fragile]
    \frametitle{Functional Programming: \texttt{arrayfun} and \texttt{cellfun}}

    \begin{enumerate}
      \item \texttt{for}-loop: Suggests sequential operations\pause
      \item \texttt{arrayfun} and \texttt{cellfun}
        \begin{enumerate}
          \item Suggest \emph{independent mutations} of array/cell elements
            \begin{lstlisting}
              x = [1 2 3 4 5];
              p = {'alpha', 'beta', 'gamma'};
            \end{lstlisting}
          \item Let you flatten out and write readable one-line \texttt{for}-loops.
        \end{enumerate}\pause
    \end{enumerate}
    \vspace{10pt}
    Example
      \begin{lstlisting}
        x = nan(5,1);
        for n = 1:5
          x(n) = sqrt(n);
        end
      \end{lstlisting}\pause
      \begin{lstlisting}
        % Alternative
        arrayfun(@sqrt, [1 2 3 4 5])
        ans =
            1.0000    1.4142    1.7321    2.0000    2.2361
      \end{lstlisting}

	\end{frame}

  \begin{frame}[fragile]
    \frametitle{Functional Programming: \texttt{arrayfun} and \texttt{cellfun}}

    Suppose that I had a function \texttt{texify} that slaps a backslash
    \texttt{\textbackslash} in front of any string passed in.\pause

    \vspace{20pt}
    \texttt{cellfun} example
      \begin{lstlisting}
        cellfun(@texify, {'alpha', 'beta', 'gamma'}, 'un', 0)
        ans =
            '\alpha'    '\beta'   '\gamma'
      \end{lstlisting}\pause
    Compare that to
      \begin{lstlisting}
        p = {'alpha', 'beta', 'gamma'};
        for n = 1:length(p)
          p{n} = texify(p{n});
        end
      \end{lstlisting}
	\end{frame}


  \begin{frame}[fragile]
    \frametitle{Functional Programming: Anonymous Functions}

    But not every function deserves an \texttt{m}-file.\pause
    \begin{lstlisting}
      function [x] = dumb(x)
        x = 2*x;
      end
    \end{lstlisting}\pause
    Can define \emph{anonymous functions}
    \begin{lstlisting}
      dumb   = @(x) 2*x;
      texify = @(s) ['\' s];
    \end{lstlisting}
    \begin{lstlisting}
      dumb(3)
      ans =
          6
    \end{lstlisting}
    \begin{lstlisting}
      texify('alpha')
      ans =
      \alpha
    \end{lstlisting}
	\end{frame}

  \begin{frame}[fragile]
    \frametitle{Functional Programming: Anonymous Functions}

    Called ``anonymous'' because you don't \emph{have} to assign them to
    anything
    \begin{lstlisting}
      cellfun(@(s) ['\' s], {'alpha', 'beta', 'gamma'}, 'un', 0);
      ans =
          '\alpha'    '\beta'   '\gamma'
    \end{lstlisting}\pause
    Another example
    \begin{lstlisting}
      fminbnd(@(x) 4*cos(x), 3, 4)
    \end{lstlisting}
	\end{frame}


  \begin{frame}[fragile]
    \frametitle{Functional Programming: Anonymous Functions}

    Finally
    \begin{enumerate}
      \item They can also \emph{inherit} things outside the namespace
        (unlike typical functions defined in \texttt{m}-files)
        \begin{lstlisting}
          >> f = 'rbny';
          >> plotDir  = '/myfolder/Plots/';
          >> saveName = @(f) [plotDir f '.pdf'];
          >> saveName('Plot1')
          ans =
          /myfolder/Plots/Plot1.pdf
        \end{lstlisting}\pause
      \item But they will still have \emph{zero} (count-it: \emph{zero})
        side-effects:
        \begin{lstlisting}
          >> x = 5
          >> dumb = @(x) 2*x;
          >> dumb(10)
          ans =
              20
          >> x
          ans =
              5
        \end{lstlisting}
    \end{enumerate}
	\end{frame}

  \begin{frame}[fragile]
    \frametitle{Functional Programming: Anonymous Functions}

    Useful inheritance:
    \begin{enumerate}
      \item Functions like \texttt{fminbnd} and \texttt{fminsearch}
        expect one-argument functions.\pause
      \item \emph{But} something like a likelihood function requires at
        least two arguments: parameters and data.\pause
      \item You can turn \emph{any} function into a one-argument
        function as follows:
        \begin{lstlisting}
      toOptimize = @(theta) myfcn(Y,theta,other,options);
        \end{lstlisting}
        This \emph{inherits} the values of \texttt{Y}, \texttt{other},
        and \texttt{options} from the namespace in which it is defined.
        They are fixed for \emph{all} values of \texttt{theta}.
        \vspace{10pt}

        Now you can run
        \begin{lstlisting}
          fminbnd(toOptimize, 0, 1);
        \end{lstlisting}
    \end{enumerate}
	\end{frame}

  \begin{frame}[fragile,shrink=10]
    \frametitle{Functional Programming: Anonymous Functions}

    Inception-level anonymous functions
    \begin{enumerate}
      \item Remember that functions are an object. They can be passed
        around, so they can \emph{also} be returned.

        Example: A call option has payoff $\max\{S-K,0\}$. Can we write
        an anonymous function that returns this payoff function?\pause
\begin{lstlisting}
  call  = @(K) (@(S) max(S-K,0));
\end{lstlisting}\pause
\begin{lstlisting}
  call5 = call(5);
\end{lstlisting}\pause
\begin{lstlisting}
  [1:10; arrayfun(call5, 1:10)]
  ans =
      1     2     3     4     5     6     7     8     9    10
      0     0     0     0     0     1     2     3     4     5
\end{lstlisting}
      \item They can also be assigned to structs
  \begin{lstlisting}
    >> tform.chg    = @(x) [NaN; diff(x)];
    >> tform.pctchg = @(x) 100*([NaN; x(2:end)./x(1:end-1)]-1);
  \end{lstlisting}\pause
  \begin{lstlisting}
    >> tform.chg([1:5]')'
    ans =
       NaN     1     1     1     1
    \end{lstlisting}
    \begin{lstlisting}
    >> tform.pctchg([1:5]')'
    ans =
        NaN  100.0000   50.0000   33.3333   25.0000
  \end{lstlisting}
      and then looped-over via dynamic referencing.

    \end{enumerate}

	\end{frame}


\section{Parallel Programming}

  \begin{frame}[fragile]
    \frametitle{Parallel Programming: The Ideal}

    Start with
    \begin{lstlisting}
    for n = 1:N
      results(n) = fcn(args{n,:});
    end
    \end{lstlisting}\pause
    Parallel version
    \begin{enumerate}
      \item You need to preallocate
      \begin{lstlisting}
        results = nan(N,1);
      \end{lstlisting}\pause
      \item You need to open up a parallel pool
      \begin{lstlisting}
        poolObject = parpool(Nworkers);
      \end{lstlisting}\pause
      \item Modify the code\pause
        \begin{lstlisting}
        parfor n = 1:N
          results(n) = fcn(args{n,:});
        end
        \end{lstlisting}\pause
      \item Close the pool after you're done
        \begin{lstlisting}
          delete(poolobj)
        \end{lstlisting}
    \end{enumerate}

	\end{frame}


  \begin{frame}[fragile]
    \frametitle{Parallel Programming: The Ideal}

    In summary, from this
    \begin{lstlisting}
    results = nan(N,1);
    for n = 1:N
      results(n) = fcn(args{n,:});
    end
    \end{lstlisting}
    To this
    \begin{lstlisting}
      results    = nan(N,1);
      poolObject = parpool(Nworkers);
      parfor n = 1:N
        results(n) = fcn(args{n,:});
      end
      delete(poolobj)
    \end{lstlisting}
	\end{frame}

  \begin{frame}[fragile]
    \frametitle{Parallel Programming: Issues}

    You want the body of a \texttt{parfor} loop to be \emph{function
    calls}
    \begin{enumerate}
      \item Indexing arguments from some numeric or cell array
      \item Storing the result in a numeric or cell array
    \end{enumerate}\pause
    Here's why
    \begin{lstlisting}
      parfor n = 1:N
        a = randn();
        results(n) = a;
      end
    \end{lstlisting}
    If you run with \texttt{N} parallel workers, Matlab will try to
    assign a random number to variable \texttt{a}, all at once,
    \texttt{N} times.

    \vspace{10pt}
    Functions get around this because each of the \texttt{N} iterations
    gets its \emph{own namespace}. There's no confusion of variable
    values across iterations.

	\end{frame}


  \begin{frame}[fragile, shrink=20]
    \frametitle{Parallel Programming: Issues}

    One annoying thing: if you run something in parallel and it errors
    out, fixing then rerunning your code will throw an error if that
    code tries to open up a parallel pool again.\pause

    \vspace{10pt}
    Here's a way around that
    \begin{lstlisting}
function [ poolobj ] = poolsetup(parworkers, varargin)
% POOLSETUP - Wrapper and easier way to set up a parallel pool.
%
% First checks if one is open already. If so, do nothing and
% just return the object for that parallel session. If not,
% open up a pool.
%
% Can also pass in files to add to the session.

  poolcheck = gcp('nocreate');
  if isempty(poolcheck)
    poolobj = parpool(parworkers); % Open up a pool
  else
    poolobj = poolcheck;
  end
  if nargin > 1
    addAttachedFiles(poolobj, varargin{1});
  end
end
    \end{lstlisting}
    This will either find your open session and use it, or start a new
    one. Use it like
    \begin{lstlisting}
      poolObject = poolsetup(parworkers); % Rather than  '= parpool(parworkers)'
    \end{lstlisting}

	\end{frame}


  \begin{frame}[fragile, shrink=20]
    \frametitle{Parallel Programming: Issues}

    Another annoying thing: One bad iteration kills the whole parallel
    loop.\pause

    \vspace{10pt}
    Here's a way around that
    \begin{lstlisting}
% ERRORPROOF - To wrap functions so that any errors don't kill a program
function [returned, errored_out] = errorproof(fcn, varargin)

try
  returned = feval(fcn, args{:});;
  errored_out = 0;
catch
  returned = lasterror();
  errored_out = 1;
end
    \end{lstlisting}
    Rewrite \texttt{parfor} loops from this
    \begin{lstlisting}
      parfor n = 1:N
        results(n) = fcn(args{n,:});
      end
    \end{lstlisting}
    to this
    \begin{lstlisting}
      parfor n = 1:N
        [results{n}, errs(n)] = errorproof(@fcn, args{n,:});
      end
    \end{lstlisting}
    This will store the last error message along an indicator as to
    whether the function errored out.

	\end{frame}

	%\frame{
			%\frametitle{Slide with Text}

			%You can use lists as usual:
			%\begin{itemize}
		%\item You can even \pause add breaks
		%\item Anytime you write ``\textbackslash pause'', two slides will
				%be generated \pause
		%\item One with everything up to pause only.
		%\item Another with everything on the slide, even
				%after pause
			%\end{itemize}
	%}



	%\frame{
			%\frametitle{Adding Some Color}

        %Maybe you want to highlight some text and really make it
        %\textcolor{red}{stand out}.
	%}

	%\frame{
			%\frametitle{Highlighting Certain Points}

        %Or maybe, you want to gray out certain text to highlight other
        %points.
        %\usebeamercolor{dull text}
        %\begin{itemize}
            %\item \alert<+>{You do that by defining the color of ``dull''
                %text and ``alerted'' text in the preamble}
            %\item \alert<+>{Then you set the color to dull text, adding
                %``\textbackslash alert\textless\texttt{+}\textgreater''
                %tags and enclosing within brackets
                %the text to be ``alert.''}
        %\end{itemize}
        %\pause
        %Each ``\textbackslash alert\textless\texttt{+}\textgreater\{text\}''
        %also adds pauses between the dull and alert
        %text.
	%}


\end{document}

